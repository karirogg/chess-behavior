\section{Introduction}
The game of chess has been among the most popular intellectual games for hundreds of years. More recently, it has been the subject of many pioneering experiments related to artificial intelligence. In the beginning, research was heavily focused on making systems that play chess well. Recently, more research has focused on how humans play chess.
In this project, we take that to an extreme. We want to develop a model that can accurately determine who is playing a given chess game, given only the moves that were played. We are repeating a previous experiment exploring this, but we have altered some methods. We explore the topic of distinguishing between expert chess players, while the previous experiment focuses mainly on amateur players.
We train a convolutional-LSTM model that generates embeddings for both players playing a chess game.
We use prototypical learning to generalize the model to new players whose data was not used in the training of the model. 

\subsection{Previous Work}
Our approach is based on previous work by
(\citealp{main_article}), where a transformer model was created from the \href{https://lichess.org/}{Lichess} dataset to determine the playing style of the player that is playing. They have adapted methods from speaker verification in the form of the Generalized end-to-end contrastive loss function introduced in (\citealp{ge2e}). Their move-level feature extraction is also based on previous developments in human-like chess-playing systems. Maia is an artificial intelligence system that is aimed towards playing chess like a human (\citealp{maia}). However, due to ethical concerns, they do not disclose their code. Therefore, we wanted to validate their results.

\subsection{Ethical Concerns}
The problem at hand is a subproblem of the more general task of identifying a decision maker from their decisions alone. Solutions to these problems reduce the possibilities of remaining anonymous online, which can be a large concern in some instances. In this particular case, a good solution would prevent players from being able to play chess online anonymously. We think this is not a substantial ethical concern, though we understand that this might trouble some people.